\section{Appendix}

We begin with proofs of the translations of the matrix $T_{\pm}^{(q)}$ into the ZX-calculus. 

\begin{proposition}\label{prop:pm_maps_zx_appendix} \textbf{/\ Propositions~\ref{prop:pm_map_q2_q4}, \ref{prop:pm_map_q3}.}
	\begin{equation*}
		\left\llbracket \ \tikzfig{pm_maps/q2} \ \right\rrbracket = \left\llbracket \ \tikzfig{pm_maps/pm} \ \right\rrbracket_{q=2} \ , \qquad
		\left\llbracket \ \tikzfig{pm_maps/q3} \ \right\rrbracket = \left\llbracket \ \tikzfig{pm_maps/pm} \ \right\rrbracket_{q=3} \ , \quad
		\left\llbracket \ \tikzfig{pm_maps/q4} \ \right\rrbracket = \left\llbracket \ \tikzfig{pm_maps/pm} \ \right\rrbracket_{q=4}
	\end{equation*}

	\begin{proof}
		Recalling $\omega = e^{i\frac{2\pi}{3}}$, the standard interpretations of phase gates in matrix form are:
		
		\begin{equation*}
			\left\llbracket \ \tikzfig{pm_maps/qubit/Z_phase} \ \right\rrbracket = 
			\begin{pmatrix}
				1 & 0 \\
				0 & e^{i\alpha}
			\end{pmatrix} \ , \quad
			\left\llbracket \ \tikzfig{pm_maps/qubit/X_phase} \ \right\rrbracket = 
			\frac{1}{2} \begin{pmatrix}
				1 + e^{i\alpha} & 1 - e^{i\alpha} \\
				1 - e^{i\alpha} & 1 + e^{i\alpha}
			\end{pmatrix} \ , \quad
			\left\llbracket \ \tikzfig{pm_maps/qutrit/Z_phase} \ \right\rrbracket = 
			\begin{pmatrix}
				1 & 0 & 0\\
				0 & e^{i\alpha} & 0 \\
				0 & 0 & e^{i\beta}
			\end{pmatrix}
		\end{equation*}
		\begin{equation*}
			\left\llbracket \ \tikzfig{pm_maps/qutrit/X_phase} \ \right\rrbracket = 
			\frac{1}{3} \begin{pmatrix}
				1 + e^{i\alpha} + e^{i\beta} & 1 + \bar{\omega}e^{i\alpha} + {\omega}e^{i\beta} & 1 + {\omega}e^{i\alpha} + \bar{\omega}e^{i\beta} \\
				1 + {\omega}e^{i\alpha} + \bar{\omega}e^{i\beta} & 1 + e^{i\alpha} + e^{i\beta} & 1 + \bar{\omega}e^{i\alpha} + {\omega}e^{i\beta} \\
				1 + \bar{\omega}e^{i\alpha} + {\omega}e^{i\beta} & 1 + {\omega}e^{i\alpha} + \bar{\omega}e^{i\beta} & 1 + e^{i\alpha} + e^{i\beta} \\
			\end{pmatrix}
		\end{equation*}

		So in the simplest case $q=2$ it is fairly straightforward to see that:

		\begin{equation}
			\left\llbracket \ \tikzfig{pm_maps/q2} \ \right\rrbracket \ = \ 
			\frac{1}{2} \begin{pmatrix}
				1 \pm i & 1 \mp i \\
				1 \mp i & 1 \pm i \\
			\end{pmatrix} \ = \ 
			\frac{\sqrt{2}}{2} e^{\mp i \frac{\pi}{4}} \begin{pmatrix}
				\pm i & 1 \\
				1 & \pm i \\
			\end{pmatrix} \ = \ 
			\frac{\sqrt{2}}{2} e^{\mp i \frac{\pi}{4}} \left\llbracket \ \tikzfig{pm_maps/pm} \ \right\rrbracket_{q=2}
		\end{equation}

		The next case $q=3$ is proved similarly:

		\begin{equation}
		\begin{aligned}
				\left\llbracket \ \tikzfig{pm_maps/q3} \ \right\rrbracket
				&= \frac{1}{3} \begin{pmatrix}
					1 + e^{\pm i\frac{2\pi}{3}} + e^{\pm i\frac{2\pi}{3}} & 1 + \bar{\omega}e^{\pm i\frac{2\pi}{3}} + {\omega}e^{\pm i\frac{2\pi}{3}} & 1 + {\omega}e^{\pm i\frac{2\pi}{3}} + \bar{\omega}e^{\pm i\frac{2\pi}{3}} \\
					1 + {\omega}e^{\pm i\frac{2\pi}{3}} + \bar{\omega}e^{\pm i\frac{2\pi}{3}} & 1 + e^{\pm i\frac{2\pi}{3}} + e^{\pm i\frac{2\pi}{3}} & 1 + \bar{\omega}e^{\pm i\frac{2\pi}{3}} + {\omega}e^{\pm i\frac{2\pi}{3}} \\
					1 + \bar{\omega}e^{\pm i\frac{2\pi}{3}} + {\omega}e^{\pm i\frac{2\pi}{3}} & 1 + {\omega}e^{\pm i\frac{2\pi}{3}} + \bar{\omega}e^{\pm i\frac{2\pi}{3}} & 1 + e^{\pm i\frac{2\pi}{3}} + e^{\pm i\frac{2\pi}{3}} \\
				\end{pmatrix} \\
				&= \frac{1}{3} \begin{pmatrix}
					\sqrt{3}e^{\pm i\frac{\pi}{2}} & \sqrt{3}e^{\mp i\frac{\pi}{6}} & \sqrt{3}e^{\mp i\frac{\pi}{6}} \\
					\sqrt{3}e^{\mp i\frac{\pi}{6}} & \sqrt{3}e^{\pm i\frac{\pi}{2}} & \sqrt{3}e^{\mp i\frac{\pi}{6}} \\
					\sqrt{3}e^{\mp i\frac{\pi}{6}} & \sqrt{3}e^{\mp i\frac{\pi}{6}} & \sqrt{3}e^{\pm i\frac{\pi}{2}} \\
				\end{pmatrix} \\
				&= \frac{\sqrt{3}}{3} e^{\mp i\frac{\pi}{6}}\begin{pmatrix}
					e^{\pm i\frac{2\pi}{3}} & 1 & 1 \\
					1 & e^{\pm i\frac{2\pi}{3}} & 1 \\
					1 & 1 & e^{\pm i\frac{2\pi}{3}} \\
				\end{pmatrix} \\
				&= \frac{\sqrt{3}}{3} e^{\mp i\frac{\pi}{6}} \left\llbracket \ \tikzfig{pm_maps/pm} \ \right\rrbracket_{q=3}
			\end{aligned}
		\end{equation}

		For the other qubit case $q=4$ we first note:

		\begin{equation}
			\left\llbracket \ \tikzfig{qubit_hadamard/yellow_box} \ \right\rrbracket = 
			\left\llbracket \ \tikzfig{qubit_hadamard/decomposed} \ \right\rrbracket =
			\left\llbracket \ \qubitZphase{\frac{\pi}{2}} \ \right\rrbracket
			\left\llbracket \ \qubitXphase{\frac{\pi}{2}} \ \right\rrbracket
			\left\llbracket \ \qubitZphase{\frac{\pi}{2}} \ \right\rrbracket = 
			\frac{1}{2\sqrt{2}} \begin{pmatrix}
				1 & 1 \\
				1 & -1 \\
			\end{pmatrix}
		\end{equation}

		Then using the standard interpretation for spiders (Definition \ref{def:qubit_standard_spiders}) we decompose the diagram in such a way that we can apply the standard interpretation:

		\begingroup
			\allowdisplaybreaks
				\begin{align*}
					\left\llbracket \ \tikzfig{pm_maps/q4} \ \right\rrbracket 
					&= \left\llbracket \ \tikzfig{pm_maps/q4/decomposed} \ \right\rrbracket \\
					&= \left(
						\left\llbracket \ \tikzfig{pm_maps/q4/id} \ \right\rrbracket \otimes 
						\left\llbracket \ \tikzfig{pm_maps/q4/pi_compare} \ \right\rrbracket
					\right)
					\left(
						\left\llbracket \ \tikzfig{pm_maps/q4/id} \ \right\rrbracket \otimes 
						\left\llbracket \ \tikzfig{pm_maps/q4/hadamard} \ \right\rrbracket \otimes 
						\left\llbracket \ \tikzfig{pm_maps/q4/id} \ \right\rrbracket 
					\right)
					\left(
						\left\llbracket \ \tikzfig{pm_maps/q4/pi_copy} \ \right\rrbracket \otimes 
						\left\llbracket \ \tikzfig{pm_maps/q4/id} \ \right\rrbracket
					\right) \\
					&= \frac{\sqrt{2}}{8} \begin{pmatrix}
						-1 & 1 & 1 & 1 \\
						1 & -1 & 1 & 1 \\
						1 & 1 & -1 & 1 \\
						1 & 1 & 1 & -1 \\
					\end{pmatrix} \\
					&= \frac{\sqrt{2}}{8} \left\llbracket \ \tikzfig{pm_maps/pm} \ \right\rrbracket_{q=4}
				\end{align*}
		\endgroup
	\end{proof}
\end{proposition}

Next we prove the local pivot equality from Theorem~\ref{thm:local_pivot_equality}. For this, recall that all the qutrit rewrite rules hold under taking adjoints, and with the roles of green and red swapped, so in particular rule $\qutritRuleEuler$ also gives:

\begin{equation}\label{eq:qutrit_hadamard_decompositions}
	\tikzfig{qutrit_rules/hadamard/euler/h} = \tikzfig{qutrit_rules/hadamard/euler/decomposition} \ , \hspace{50pt} 
	\tikzfig{qutrit_rules/hadamard/euler/h} = \tikzfig{hadamard_lemmas/decompositions/h} \ , \hspace{50pt} 
	\tikzfig{hadamard_lemmas/decompositions/h_dagger} = \tikzfig{hadamard_lemmas/decompositions/h_dagger_1} \ , \hspace{50pt}
	\tikzfig{hadamard_lemmas/decompositions/h_dagger} = \tikzfig{hadamard_lemmas/decompositions/h_dagger_2} \
\end{equation}

\begin{theorem}\label{thm:local_pivot_equality_appendix} \textbf{/\ Theorem~\ref{thm:local_pivot_equality}.} 
	Given $a \in \mathbb{Z}_3$ and a graph state $(G, W)$ containing connected nodes $i$ and $j$, define the following:
	\begin{itemize}
		\item $N_{=}(i, j) \defeq \left\{k \in N(i) \cap N(j) \mid w_{k,i} = w_{k,j} \right\}$
		\item $N_{\neq}(i, j) \defeq \left\{k \in N(i) \cap N(j) \mid w_{k,i} \neq w_{k,j} \right\}$
	\end{itemize} 
	Then the following equation relates $G$ and its proper $a$-local pivot along $ij$:
	\ctikzfig{graph_state/proper_local_pivot}
	\begin{proof}
		Annoyingly, proving this in full generality in one go - i.e. for a proper $a$-local pivot along an edge $ij$ of weight $b$ - becomes a bit tricky diagramatically, because it becomes hard to keep track of all the variable edge weights. Fortunately the four cases ($a, b \in \{1,2\}$) split into two pairs of symmetric cases: $a = b$ and $a \neq b$.\newline

		Now, it suffices to only draw a fragment of a graph state. Certainly we consider nodes $i$ and $j$ and the edge $ij$ between them. Then define $N_x^y$ to be the set $\left\{k \mid w_{k,i} = x, w_{k,j} = y \right\}$, for $x, y, \in \mathbb{Z}_3$. We will consider a representative node $k_x^y$ from each $N_x^y \neq N_0^0$, as well as its edges $ik_x^y$ and $jk_x^y$. All other nodes and edges are irrelevant; this is because we are only interested in nodes and edges that \textit{affect} the three local complementation operations on $i$ and $j$ - we aren't concerned with those that are only \textit{affected by} the operations.\newline

		So for the case $a=b$, we show the proper $1$-local pivot along $ij$ of weight $1$:

		\begingroup
			\allowdisplaybreaks
			\setlength{\jot}{20pt}
			\begin{align*}
				&\ &&\tikzfig{proper_1_local_pivot/weight_1/step_1} 
				&&&\xeq{\ref{thm:local_comp_equality}} 
				&&&&\tikzfig{proper_1_local_pivot/weight_1/step_2} \\
				&\xeq{\ref{thm:local_comp_equality}} 
				&&\tikzfig{proper_1_local_pivot/weight_1/step_3} 
				&&&\xeq{\ref{thm:local_comp_equality}} 
				&&&&\tikzfig{proper_1_local_pivot/weight_1/step_4} \\
				&\xeqq{\eqref{eq:qutrit_hadamard_decompositions}}{\qutritRuleFusion} 
				&&\tikzfig{proper_1_local_pivot/weight_1/step_5} 
				&&&= &&&&\tikzfig{proper_1_local_pivot/weight_1/step_6} \\
			\end{align*}
		\endgroup

		The story is similar for the case $a \neq b$; here we show the proper $1$-local pivot along $ij$ of weight $2$:

		\begingroup
			\allowdisplaybreaks
			\setlength{\jot}{20pt}
			\begin{align*}
				&\ &&\tikzfig{proper_1_local_pivot/weight_2/step_1} 
				&&&\xeq{\ref{thm:local_comp_equality}} 
				&&&&\tikzfig{proper_1_local_pivot/weight_2/step_2} \\
				&\xeq{\ref{thm:local_comp_equality}} 
				&&\tikzfig{proper_1_local_pivot/weight_2/step_3} 
				&&&\xeq{\ref{thm:local_comp_equality}} 
				&&&&\tikzfig{proper_1_local_pivot/weight_2/step_4} \\
				&\xeqq{\eqref{eq:qutrit_hadamard_decompositions}}{\qutritRuleFusion} 
				&&\tikzfig{proper_1_local_pivot/weight_2/step_5} 
				&&&= &&&&\tikzfig{proper_1_local_pivot/weight_2/step_6} \\
			\end{align*}
		\endgroup

		% \ctikzfig{proper_1_local_pivot_weight_2_line_1}
		% \ctikzfig{proper_1_local_pivot_weight_2_line_2}
		% \ctikzfig{proper_1_local_pivot_weight_2_line_3}

		The case $a=b=2$ is the same as $a=b=1$, except with the roles of purple and blue edges interchanged, so by symmetry of the diagram the only difference is in the phase gates added to the outputs. Namely, on each phase gate we replace all instances of $1$ with a $2$. Thus the roles of $H$ and $H^\dagger$ are swapped too. Likewise for the case $(a,b) = (2,1)$ with respect to the case $(a,b) = (1,2)$.

	\end{proof}
\end{theorem}

Now we prove the three elimination theorems for $\mathcal{M}$-, $\mathcal{N}$- and \Pspiders. First, we require two lemmas.

\begin{lemma}\label{lem:leg_flip}
	The following `leg flip' equation holds in the qutrit ZX-calculus. Moreover, it holds with the roles of green and red swapped.
	\begin{equation*}
		\tikzfig{leg_flip/1} = \tikzfig{leg_flip/5}
	\end{equation*}
	\begin{proof}
		\begin{equation*}
			\tikzfig{leg_flip/1} \ \xeq{\qutritRuleFusion} \ 
			\tikzfig{leg_flip/2} \ \xeq{\qutritRuleSnake} \ 
			\tikzfig{leg_flip/3} \ \xeq{\qutritRuleColourChange} \  
			\tikzfig{leg_flip/4} \ \xeq{\qutritRuleColourChange} \  
			\tikzfig{leg_flip/5}
		\end{equation*}
	\end{proof}
\end{lemma}

\begin{lemma}\label{lem:substantial_m_copy}
	The following more substantial `$\mathcal{M}$-copy' rule holds in the qutrit ZX-calculus, for any \Mspider\ state (i.e. $m \in \{0, 1, 2\}$ below). Moreover, it holds with the roles of green and red swapped.
	\begin{equation*}
		\tikzfig{m_copies/full/1} = \tikzfig{m_copies/full/4}
	\end{equation*}
	\begin{proof}
		First we prove that an \Mspider\ with non-trivial phase (i.e. $m \in \{1, 2\}$) satisfies a copy rule exactly like the rule $\qutritRuleZeroCopy$:
		\begin{equation}\label{eq:non_trivial_m_copy}
			\tikzfig{m_copies/part_1/1} \ \xeq{\qutritRuleFusion} \ 
			\tikzfig{m_copies/part_1/2} \ \xeq{\qutritRuleMCopy} \ 
			\tikzfig{m_copies/part_1/3} \ \xeq{\qutritRuleZeroCopy} \ 
			\tikzfig{m_copies/part_1/4} \ \xeq{\qutritRuleFusion} \ 
			\tikzfig{m_copies/part_1/5}
		\end{equation}
		Then we prove the case $\alpha = \beta = 0$ by induction:
		\begin{equation}\label{eq:m_copy_through_trivial}
			\tikzfig{m_copies/part_2/1} \ \xeq{\qutritRuleFusion} \ 
			\tikzfig{m_copies/part_2/2} \ \xeq{\eqref{eq:non_trivial_m_copy}} \ 
			\tikzfig{m_copies/part_2/3} \ \xeq{\text{ind.}} \ 
			\tikzfig{m_copies/part_2/4}
		\end{equation}
		Which finally allows us to prove the full statement, where the last equality is just dropping the scalar term:
		\begin{equation*}
			\tikzfig{m_copies/full/1} \ \xeq{\qutritRuleFusion} \ 
			\tikzfig{m_copies/full/2} \ \xeq{\eqref{eq:m_copy_through_trivial}} \ 
			\tikzfig{m_copies/full/3} \ = \
			\tikzfig{m_copies/full/4}
		\end{equation*}
	\end{proof}
\end{lemma}

We are now in a position to prove the \Mspider\ elimination theorem.

\begin{theorem}\label{thm:eliminate_M_spiders_appendix} \textbf{/\ Theorem~\ref{thm:eliminate_M_spiders}.}
	Given any graph-like ZX-diagram containing two interior \Mspiders\ $i$ and $j$ connected by edge $ij$ of weight $w_{i,j} \eqdef w \in \{1,2\}$, suppose we perform a proper $\pm w$-local pivot along $ij$. Then the new ZX-diagram is related to the old one by the equality:
	
	\ctikzfig{eliminate_M_spiders/theorem_LHS}
	\ctikzfig{eliminate_M_spiders/theorem_RHS}
	
	where all changes to weights of edges where neither endpoint is $i$ or $j$ are omitted. Furthermore, in order to save space, each node with phase \qutritZphase{a_x^y}{b_x^y} is representative of \textit{all} nodes connected to $i$ by an $x$-weighted edge and to $j$ by a $y$-weighted edge.

	\begin{proof}
		We show the case where $w_{ij} \eqdef w = 1$, with the case $w = 2$ being completely analogous. We can choose either a proper $1$-local pivot or a proper $2$-local pivot; both give the same result. Here we only show the former:
		\begingroup
			\allowdisplaybreaks
			\setlength{\jot}{30pt}
			\begin{align*}
				&\ &&\tikzfig{eliminate_M_spiders/step_1} \\
				&\xeq{\qutritRuleFusion} 
				&&\tikzfig{eliminate_M_spiders/step_2} \\
				&\xeq{\ref{thm:local_pivot_equality_appendix}} 
				&&\tikzfig{eliminate_M_spiders/step_3} \\
				&\xeq{\qutritRuleColourChange}
				&&\tikzfig{eliminate_M_spiders/step_4} \\
				&\xeq{\ref{lem:leg_flip}}
				&&\tikzfig{eliminate_M_spiders/step_5} \\
				&\xeq{\ref{lem:substantial_m_copy}}
				&&\tikzfig{eliminate_M_spiders/step_6} \\
				&= 
				&&\tikzfig{eliminate_M_spiders/step_7} \\
				&\xeq{\qutritRuleColourChange}
				&&\tikzfig{eliminate_M_spiders/step_8} \\
				&\xeq{\qutritRuleFusion}
				&&\tikzfig{eliminate_M_spiders/step_9} \\
			\end{align*}
		\endgroup
	\end{proof}
\end{theorem}

Proving the corresponding \Nspider\ elimination theorem requires a further lemma, which allows us to turn an \Nspider\ state of one colour into an \Nspider\ state of the other.

\begin{lemma}\label{lem:N_state_colour_change}
	The following rules hold in the qutrit ZX-calculus:
	\begin{equation*}
		\tikzfig{n_states/0_1/z} \ = \ \tikzfig{n_states/0_1/x} \ , \qquad
		\tikzfig{n_states/0_2/z} \ = \ \tikzfig{n_states/0_2/x} \ , \qquad
		\tikzfig{n_states/1_0/z} \ = \ \tikzfig{n_states/1_0/x} \ , \qquad
		\tikzfig{n_states/2_0/z} \ = \ \tikzfig{n_states/2_0/x}
	\end{equation*}
	\begin{proof}
		The key observation is that for any green \Nspider\ state with phase $\qutritZphase{n}{n'}$, we have a choice of two colour change rules which we could use to turn it into a red \Nspider\ state with a $H$- or $H^\dagger$-box on top:
		
		\begin{equation*}
			\tikzfig{n_states/general/x_n_n'} \ \xeq{\qutritRuleColourChange} \ 
			\tikzfig{n_states/general/z_n_n'} \ \xeq{\qutritRuleColourChange} \ 
			\tikzfig{n_states/general/x_n'_n}
		\end{equation*}

		Of these two choices, exactly one has a decomposition of of the $H$-/$H^\dagger$-box as in \eqref{eq:qutrit_hadamard_decompositions} that allows the bottom two red spiders to fuse into an \Mspider, which we can then move past the green spider above it via \ref{lem:substantial_m_copy}. For brevity we only show the case $\qutritZphase{n}{n'} = \qutritZphase{0}{1}$:

		\begin{equation*}
			\tikzfig{n_states/0_1/1} \ \ \xeq{\qutritRuleColourChange} \ \ 
			\tikzfig{n_states/0_1/2} \ \ \xeq{\eqref{eq:qutrit_hadamard_decompositions}} \ \ 
			\tikzfig{n_states/0_1/3} \ \ \xeq{\qutritRuleFusion} \ \ 
			\tikzfig{n_states/0_1/4} \ \ \xeq{\ref{lem:substantial_m_copy}} \ \ 
			\tikzfig{n_states/0_1/5} \ \ \xeq{\qutritRuleFusion} \ \ 
			\tikzfig{n_states/0_1/6}
		\end{equation*}
	\end{proof}
\end{lemma}

\begin{corollary}\label{cor:N_effect}
	The following equations hold in the qutrit ZX-calculus:
	\begin{equation*}
		\tikzfig{n_states/corollary/0_n/lhs} \ = \ \tikzfig{n_states/corollary/0_n/rhs} \ , \qquad
		\tikzfig{n_states/corollary/n_0/lhs} \ = \ \tikzfig{n_states/corollary/n_0/rhs}
	\end{equation*}
	\begin{proof}
		Again we only prove one case, the rest being analogous. Each case uses \ref{lem:N_state_colour_change} in its adjoint form - recall that the adjoint of a spider is found by swapping inputs and outputs and negating angles.
		\begin{equation*}
			\tikzfig{n_states/corollary/0_1/1} \ \ = \ \ 
			\tikzfig{n_states/corollary/0_1/2} \ \ \xeq{\ref{lem:N_state_colour_change}} \ \ 
			\tikzfig{n_states/corollary/0_1/3} \ \ \xeq{\qutritRuleFusion} \ \ 
			\tikzfig{n_states/corollary/0_1/4} \ \ = \ \ 
			\tikzfig{n_states/corollary/0_1/5}
		\end{equation*}
	\end{proof}
\end{corollary}

\begin{theorem}\label{thm:eliminate_N_spiders_appendix} \textbf{/\ Theorem~\ref{thm:eliminate_N_spiders}.}
	Given any graph-like ZX-diagram containing an interior \Nspider\ $k$ with phase \qutritZphase{0}{n} for $n \in \{1,2\}$, suppose we perform a $(-n)$-local complementation at $k$. Then the new ZX-diagram is related to the old one by the equality:

	\begin{equation*}
		\tikzfig{eliminate_N_spiders/0_n/step_1} \quad = \quad \tikzfig{eliminate_N_spiders/0_n/step_9}
	\end{equation*}

	where all changes to weights of edges where neither endpoint is $k$ are omitted. If instead $k$ has phase \qutritZphase{n}{0} for $n \in \{1,2\}$, suppose we perform the same $(-n)$-local complementation at $k$. Then the equality relating the new and old diagrams becomes:

	% Spiders with phases \qutritZphase{a_1}{b_1} ... \qutritZphase{a_r}{b_r} are all the neighbours of $k$ connected by a $1$-weighted (blue) edge, while spider with phases \qutritZphase{c_1}{d_1} ... \qutritZphase{c_s}{d_s} are all the neighbours of $k$ connected by a $2$-weighted (purple) edge.\newline

	\begin{equation*}
		\tikzfig{eliminate_N_spiders/n_0/step_1} \quad = \quad \tikzfig{eliminate_N_spiders/n_0/step_9}
	\end{equation*}

	\begin{proof}
		We prove the case where $k$ has phase \qutritZphase{0}{n} for $n \in \{1,2\}$, the other case being near-identical.
		\begingroup
			\allowdisplaybreaks
			\setlength{\jot}{20pt}
				\begin{align*}
					&\ &&\tikzfig{eliminate_N_spiders/0_n/step_1} 
					&&&\xeq{\qutritRuleFusion} 
					&&&&\tikzfig{eliminate_N_spiders/0_n/step_2} \\
					&\xeq{\ref{thm:local_comp_equality}} 
					&&\tikzfig{eliminate_N_spiders/0_n/step_3} 
					&&&\xeqq{\ref{cor:N_effect}}{\qutritRuleFusion} 
					&&&&\tikzfig{eliminate_N_spiders/0_n/step_4} \\
					&\xeq{\ref{lem:leg_flip}} 
					&&\tikzfig{eliminate_N_spiders/0_n/step_5} 
					&&&\xeq{\ref{lem:substantial_m_copy}} 
					&&&&\tikzfig{eliminate_N_spiders/0_n/step_6} \\
					&\xeq{\eqref{eq:qutrit_dashed_lines}}
					&&\tikzfig{eliminate_N_spiders/0_n/step_7} 
					&&&\xeq{\qutritRuleColourChange} 
					&&&&\tikzfig{eliminate_N_spiders/0_n/step_8} \\
					&\xeq{\qutritRuleFusion} 
					&&\tikzfig{eliminate_N_spiders/0_n/step_9} \\
				\end{align*}
		\endgroup
	\end{proof}
\end{theorem}

Similarly the corresponding \Pspider\ elimination theorem requires a lemma allowing us to turn a \Pspider\ state of one colour into a \Pspider\ state of the other. As above, we will use this lemma its adjoint form in the proof of the main theorem.

\begin{lemma}\label{lem:P_state_colour_change}
	The following rule holds in the qutrit ZX-calculus:
	\begin{equation*}
		\tikzfig{p_state/1} \ = \ \tikzfig{p_state/6}
	\end{equation*}
	\begin{proof}
		The proof structure is exactly as in \ref{lem:N_state_colour_change}, only for \Pspiders\ it's even simpler:
		\begin{equation*}
			\tikzfig{p_state/1} \ \ \xeq{\qutritRuleColourChange} \ \ 
			\tikzfig{p_state/2} \ \ \xeq{\eqref{eq:qutrit_hadamard_decompositions}} \ \ 
			\tikzfig{p_state/3} \ \ \xeq{\qutritRuleFusion} \ \ 
			\tikzfig{p_state/4} \ \ \xeq{\ref{lem:substantial_m_copy}} \ \ 
			\tikzfig{p_state/5} \ \ \xeq{\qutritRuleFusion} \ \ 
			\tikzfig{p_state/6}
		\end{equation*}
	\end{proof}
\end{lemma}

\begin{theorem}\label{thm:eliminate_P_spiders_appendix} \textbf{/\ Theorem~\ref{thm:eliminate_P_spiders}.} 
	Given any graph-like ZX-diagram containing an interior \Pspider\ $k$ with phase \qutritZphase{p}{p} for $p \in \{1,2\}$, suppose we perform a $p$-local complementation at $k$. Then the new ZX-diagram is related to the old one by the equality:

	\begin{equation*}
		\tikzfig{eliminate_P_spiders/step_1} \quad = \quad \tikzfig{eliminate_P_spiders/step_9}
	\end{equation*}

	where all changes to weights of edges where neither endpoint is $k$ are omitted. 

	\begin{proof}
		\begingroup
			\allowdisplaybreaks
			\setlength{\jot}{20pt}
				\begin{align*}
					&\ &&\tikzfig{eliminate_P_spiders/step_1} 
					&&&\xeq{\qutritRuleFusion} 
					&&&&\tikzfig{eliminate_P_spiders/step_2} \\
					&\xeq{\ref{thm:local_comp_equality}} 
					&&\tikzfig{eliminate_P_spiders/step_3} 
					&&&\xeqq{\ref{lem:P_state_colour_change}}{\qutritRuleFusion} 
					&&&&\tikzfig{eliminate_P_spiders/step_4} \\
					&\xeq{\ref{lem:leg_flip}} 
					&&\tikzfig{eliminate_P_spiders/step_5} 
					&&&\xeq{\ref{lem:substantial_m_copy}} 
					&&&&\tikzfig{eliminate_P_spiders/step_6} \\
					&\xeq{\eqref{eq:qutrit_dashed_lines}}
					&&\tikzfig{eliminate_P_spiders/step_7} 
					&&&\xeq{\qutritRuleColourChange} 
					&&&&\tikzfig{eliminate_P_spiders/step_8} \\
					&\xeq{\qutritRuleFusion} 
					&&\tikzfig{eliminate_P_spiders/step_9} \\
				\end{align*}
		\endgroup
	\end{proof}
\end{theorem}