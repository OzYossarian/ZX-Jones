\section{Introduction}



The ZX-calculus originates in
quantum foundations \cite{Coecke2011}.
It is a graphical language that allows for reasoning about ZX-diagrams, which are composed of primitive generators \cite{vandewetering2020zxcalculus} called `spiders'.
A ZX-diagram is interpreted as a tensor network over
a (semi)ring.
The calculus
is defined by a set of rewrite rules involving spiders. Performing a rewrite amounts to a
subdiagram substitution, thus transforming the diagram.
Note that the set of rewrite rules \emph{depends} on the semantics, i.e. the (semi)ring over which the spiders are interpreted as tensors.
Importantly, ZX rewrites are sound and complete for linear maps over the chosen (semi)ring.


In the context of quantum computing,
the Gottesman-Knill theorem states that stabilizer quantum
circuits can be
simulated \emph{efficiently} on classical computers \cite{Aaronson2004},
where by simulation here we mean the \emph{exact} computation of quantum probability amplitudes.
An alternative - and in fact more general - proof of the Gottesman-Knill theorem for qubits has been obtained graphically in terms of the qubit ZX-calculus.
Stabilizer circuits can be expressed by the \emph{stabilizer}
fragment of the ZX-calculus,
and stabilizer ZX-diagrams can be rewritten efficiently
in terms of derived rewrite strategies \cite{graph_theoretic_simplification}.
Moreover, these simplifying rewrites can be viewed as theorems
that are useful for simplifying or simulating universal quantum circuits, not just stabilizer circuits.
The key rewrite rules that enable the efficient rewriting
of qubit stabilizer ZX-diagrams
are called \emph{local complementation} and \emph{pivot},
the latter being a composition of three of the former \cite{graph_theoretic_simplification}.
In this work, we recall the \emph{qutrit} version of local complementation \cite{harny_completeness} and derive the corresponding pivot rule.
We then derive simplification strategies
which allow for the \emph{efficient} rewriting of qutrit stabilizer ZX-diagrams,
the core result of this work.



Interestingly, though the motivation for studying ZX diagrams stems from quantum computation, they have broader applicability as they
can express arbitrary linear maps;
every quantum circuit is a ZX-diagram but not vice versa.
A plethora of hard problems in physics and computer science
regard interacting many-body multi-state systems - both classical and quantum -
and reduce to \emph{exactly} computing a single scalar.
These range from quantum amplitudes,
partition functions in classical statistical mechanics,
counting problems, probabilistic inference, and many more.
Problem instances can be encoded as closed
tensor networks over an appropriate (semi)ring.
The scalar can be evaluated by \emph{full tensor contraction},
which in general is \#P-hard \cite{Damm2002},
and finding the optimal contraction path for a general tensor network is NP-hard.


If a scalar of interest is expressed as a closed ZX-diagram,
one can rewrite the diagram by applying rewrite rules; one's goal is then to perform \emph{full diagram simplification} in order to compute the desired scalar.
Again, simplifying arbitrary closed ZX-diagrams is \#P-hard and finding the optimal rewrite strategy is NP-hard.
Note that
given an arbitrary tensor network, one can attempt to invent rewrite rules by inspecting the contents of the tensors and performing linear-algebra operations \cite{gray2020hyperoptimized}.


ZX-diagrams have been leveraged to study the complexity of boolean satisfiability (SAT) and model counting (\#SAT) \cite{debeaudrap2020tensor}.
Model counting entails computing the number of satisfying assignments to a boolean formula,
% (given in conjunctive normal form)
and this number can be expressed as a closed diagram.
Specifically, one finds that in the case of XORSAT and \#XORSAT,
known to be tractable,
the corresponding ZX-diagrams representing the problems exactly correspond to qubit stabilizer diagrams.
Going further, the ZH calculus \cite{backens2018zh}, an extension of the ZX-calculus, can be imported to make graphically obvious
why 2SAT is in P but \#2SAT is \#P-complete, whereas 3SAT and \#3SAT are both hard.
The key realisation is that the decision problem is in fact a counting problem
but where the diagram is interpreted over the Boolean semiring.
Then the \emph{rewrite rules change accordingly} and quickly imply the above result by allowing efficient simplification strategies.

%Such observations make apparent the unifying power of diagrammatic reasoning for studying the complexity of problem families.  
% of universal interest,
% where the degrees of freedom are two-dimensional.
Continuing in the spirit of \cite{debeaudrap2020tensor}, we treat the ZX-calculus as a library of rewrite rules, from which we import the appropriate tools for the job depending on the occasion.
Here we present two case studies: evaluating the Jones polynomial at lattice roots of unity, and graph colouring.
Both of these problem families reduce to evaluating
closed tensor networks and show a transition in complexity
at a particular dimension, below which the tensor network corresponds to a stabilizer ZX-diagram.
We underline that throughout this work, all rewrites are valid
\emph{up to a scalar}
since we are only concerned about making statements about complexity.
Keeping track of multiplicative scalar factors that arise under diagram rewriting can be done efficiently.
%We are thus justified in omitting them for the purposes of this work.We also emphasise that the qutrit `elimination' rewrite rules we derive in Section \ref{sec:qutrits}\ have independent ramifications for circuit optimisation, though these are not explored here.
