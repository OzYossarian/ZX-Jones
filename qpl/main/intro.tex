\section{Introduction}


A plethora of hard problems of interest to physics and computer science
regard interacting many-body multi-state systems, classical or quantum,
and reduce to \emph{exactly} computing a single scalar.
These range from probability amplitudes in quantum mechanics,
partition functions in classical statistical mechanics,
counting problems, probabilistic inference, and many more.
Problem instances can be encoded as a \emph{closed}
tensor-networks over an appropriate (semi)ring.
The scalar can be evaluated by \emph{full tensor contraction},
which in general is \#P-hard \cite{Damm2002}.
In fact, finding the optimal contraction path for a general tensor network is NP-hard.

The ZX calculus is a graphical language whose origin lies in the field of quantum foundations \cite{Coecke2011}.
It allows for reasoning about ZX diagrams, i.e. tensor networks which are expressed in terms of primitive generators \cite{vandewetering2020zxcalculus}.
The generators have a concrete representation as tensors over
a (semi)ring and they obey a set of rewrite rules which respect semantics \cite{wang2020completeness}.
Importantly, the rewrite rules \emph{depend} on the (semi)ring over which the diagram is interpreted as a tensor network.
If a scalar of interest is expressed as a ZX diagram,
one can rewrite the diagram by applying rewrite rules.
One's goal is then to apply a sequence of rewrites and perform \emph{full diagram simplification} in order to compute the desired scalar.
%Again, simplifying arbitrary closed ZX diagrams is \#P-hard and finding the optimal rewrite strategy is NP-hard.
Note that
given an arbitrary tensor network, one can attempt to invent rewrite rules by inspecting the contents of the tensors and performing linear-algebra operations \cite{gray2020hyperoptimized}.

In the context of quantum computing,
the Gottesman-Knill theorem states that stabilizer (or Clifford) circuits can be
simulated \emph{efficiently} on classical computers \cite{Aaronson2004}.
By simulation here we mean the \emph{exact} computation of \emph{amplitudes}.
These circuits are can be expressed by the \emph{stabilizer} fragment of the ZX calculus.
An alternative, and in fact more general, proof of the Gottesman-Knill theorem for qubits has been obtained graphically in terms of the qubit ZX calculus.
Interestingly, even if the motivation for studying stabilizer diagrams stems from quantum computation, the result that stabilizer ZX diagrams can be simplified efficiently has broader applicability. This is because
ZX diagrams can express arbitrary linear maps; not every ZX diagram is a quantum circuit, but every quantum circuit can be cast as a ZX diagram.

Beyond quantum computing, ZX diagrams have been leveraged to study the complexity of boolean satisfiability (SAT) and model counting (\#SAT) \cite{debeaudrap2020tensor}.
Model counting entails computing the number of satisfying assignments to a boolean formula (which is given in conjunctive normal form)
and this number can be expressed a closed diagram.
Specifically, one finds that in the case of XORSAT and \#XORSAT,
which are known to be tractable,
the corresponding ZX diagrams representing the problems exactly correspond to qubit stabilizer diagrams, which are efficient to simplify.
Going further, the ZH calculus \cite{backens2018zh}, an extension of the ZX calculus, can be imported to make graphically obvious
why 2SAT is in P but \#2SAT is \#P-complete while 3SAT is NP-complete and \#3SAT is \#P-hard.
That is, the counting version of 2SAT is hard while deciding is easy, but this is not the case for 3SAT, where both deciding and counting are hard.
The \emph{key realisation} is in that the decision problem is a counting problem
but where the diagram is interpreted over the Boolean semiring.
% (where $1+1=2$).
Then the \emph{rewrite rules change accordingly} and obviously imply the above result by allowing efficient simplification strategies.

Such observations make apparent the \emph{unifying power of diagrammatic reasoning
for studying the complexity} of problem families of universal interest,
where the degrees of freedom are two-dimensional.
Building on the spirit of \cite{debeaudrap2020tensor}, we treat ZX as a library comprising out-of-the-box and readily available diagrammatic rewrite rules from which we import the appropriate tools for the job depending on the occasion.
The key rewrite rules that enable the efficient diagram simplification
of qubit stabilizer ZX diagrams
are called \emph{local complementation} and \emph{pivot},
the latter being composition of three of the former \cite{graph_theoretic_simplification}.
In this work, we recall the qutrit version of local complementation and derive the corresponding pivot rule
which implies that qutrit stabilizer diagrams can be simplified efficiently.
We then present two case studies: evaluating the Jones polynomial at lattice roots of unity, and graph colouring.
Both of these problem families reduce to evaluating
closed tensor networks and show a transition in complexity
at a particular dimension, below which the tensor network corresponds to a stabilizer ZX diagram.

We underline that throughout this work, all rewrites are valid
\emph{up to scalar}
since we are only concerned about making statements about complexity.
Importantly, note that keeping track of multiplicative scalar factors that arise under diagram rewriting can be done efficiently {\bf [ToDo: ref]}.
We are thus justified in omitting them for the purposes of this work.
