\section{Discussion and Outlook}

In this work, we leveraged local complementation and pivot operations to flesh out non-trivial simplifications strategies for qutrit ZX diagrams. We expect these results to have immediate uses in qutrit circuit optimisation. 

Somewhat tangentially, we have also used our new simplification strategies to provide complexity-theoretic insight into certain tensor network problems that demonstrate a `step-change' in complexity when the dimension $d$ carried by the wires is greater than a particular value. When such problems are translated into the ZX-calculus, this step-change corresponds to whether a diagram is inside or outside of the efficiently reducible stabilizer fragment of the calculus.

% When the solution to a problem is encoded as a closed ZX-diagram,
% this solution can be obtained via full diagram simplification by applying the rules of the calculus.
% The problem can be solved \emph{efficiently} when the diagram encoding its answer is in the stabilizer fragment of the calculus.

% We also looked at two case studies.
% We reviewed complexity results on evaluating the Jones polynomial and counting graph colourings, two problems which can be cast in tensor network form.
% We then recovered known complexity results using only the language of the ZX-calculus.

A main path for future work entails the generalisation of our stabilizer simplification rules to qudits of higher dimensions. Recent work on the completeness of the stabilizer fragment of the qudit ZX-calculus for odd prime $d$ should provide insight and motivation, as well as some very elegant simplifications to the calculus \cite{Booth_2022}. Among these simplifications are a restoration of the maxim `only topology matters' \cite{Carette_2021}, and a representation of any stabilizer phase as a tuple $(x, y) \in \mathbb{Z}_d \times \mathbb{Z}_d$. If nothing else, it would be useful to rederive the qutrit results given here in this language - had Carette's excellent work \cite{Carette_2021} appeared before starting this project, we would certainly have used from the outset the `flexsymmetric' version of the ZX-calculus definied therein, and put to good use in other recent work \cite{van_de_Wetering_2022}. 

On a related note, Ref. \cite{Booth_2022} concludes by asking about extending their stabilizer fragment completeness results to non-prime dimensions $d$. In one of our Master's theses, we defined a parametrisation of stabilizer phases as tuples $(x, y)$ from some group isomorphic to $\mathbb{Z}_d \times \mathbb{Z}_d$ valid for \emph{all} dimensions $d$, not just odd prime ones \cite[Theorem 5.2]{TeagueMasters}. Though finding such a parametrisation is not at all the main obstacle to extending these completeness results to non-prime dimensions, we feel we should mention this result here in case it can play even a small role in this endeavour, as we have not seen it appear anywhere else in the literature.

The other primary direction of further research is in circuit extraction \cite{backens2020again} for qudit circuits; one could hope to find graph-theoretic simplification \cite{graph_theoretic_simplification} strategies analogous to those for the case of qubits.

% A final - somewhat trivial - direction regards defining scalar-exact versions of these rewrite rules, which would be necessary for concrete applications to problems where $d$-state systems are the native degrees of freedom.
% Finally, it would be interesting to investigate
% the expressive power of generalisations of ZX-calculus
% in terms of what problems it can encode.