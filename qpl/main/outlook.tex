\section{Outlook}

In this work, we fleshed out non-trivial simplifications strategies for qutrit ZX diagrams
and
we have advocated for the use of the ZX-calculus
as a unifying graphical framework for treating a broad range of many-body problems on equal footing, and for diagnosing the complexity of interesting families of problems.
% When the solution to a problem is encoded as a closed ZX-diagram,
% this solution can be obtained via full diagram simplification by applying the rules of the calculus.
% The problem can be solved \emph{efficiently} when the diagram encoding its answer is in the stabilizer fragment of the calculus.

% We also looked at two case studies.
% We reviewed complexity results on evaluating the Jones polynomial and counting graph colourings, two problems which can be cast in tensor network form.
% We then recovered known complexity results using only the language of the ZX-calculus.

A main path for future work entails the generalisation
of our stabilizer simplification rules to qudits of prime dimensions.
This in turn also motivates work in circuit extraction \cite{backens2020again} for qudit circuits; one could hope to find graph-theoretic simplification \cite{graph_theoretic_simplification} strategies analogous to those for the case of qubits.
Another direction regards defining scalar-exact versions of the rewrite rules, which would be necessary for concrete applications to problems where $d$-state systems are the native degrees of freedom.
% Finally, it would be interesting to investigate
% the expressive power of generalisations of ZX-calculus
% in terms of what problems it can encode.