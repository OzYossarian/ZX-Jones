\section{$\bf{\pm}$-Boxes in the ZX Calculus}

We close by proving that the $\pm$-boxes in \eqref{eq:pm_tensor} correspond to stabilizer ZX-diagrams.  

\begin{proposition}\label{prop:pm_maps_zx_appendix} \textbf{/\ Propositions~\ref{prop:pm_map_q2_q4}, \ref{prop:pm_map_q3}.}
	The following equalities hold up to a scalar under the standard interpretation:
	\begin{equation*}
		\left\llbracket \ \tikzfig{pm_maps/pm} \ \right\rrbracket_{d=2} \simeq 
		\left\llbracket \ \tikzfig{pm_maps/q2} \ \right\rrbracket ~, 
		\hspace{30pt}
		\left\llbracket \ \tikzfig{pm_maps/pm} \ \right\rrbracket_{d=3} \simeq
		\left\llbracket \ \tikzfig{pm_maps/q3} \ \right\rrbracket ~,
		\hspace{30pt}
		\left\llbracket \ \tikzfig{pm_maps/pm} \ \right\rrbracket_{d=4} \simeq 
		\left\llbracket \ \tikzfig{pm_maps/q4} \ \right\rrbracket
	\end{equation*}

	\begin{proof}
		Recalling $\omega = e^{i\frac{2\pi}{3}}$, the standard interpretations of phase gates in matrix form are:
		
		\begin{equation*}
			\left\llbracket \ \tikzfig{pm_maps/qubit/Z_phase} \ \right\rrbracket = 
			\begin{pmatrix}
				1 & 0 \\
				0 & e^{i\alpha}
			\end{pmatrix} \ , \quad
			\left\llbracket \ \tikzfig{pm_maps/qubit/X_phase} \ \right\rrbracket = 
			\frac{1}{2} \begin{pmatrix}
				1 + e^{i\alpha} & 1 - e^{i\alpha} \\
				1 - e^{i\alpha} & 1 + e^{i\alpha}
			\end{pmatrix} \ , \quad
			\left\llbracket \ \tikzfig{pm_maps/qutrit/Z_phase} \ \right\rrbracket = 
			\begin{pmatrix}
				1 & 0 & 0\\
				0 & e^{i\alpha} & 0 \\
				0 & 0 & e^{i\beta}
			\end{pmatrix}
		\end{equation*}
		\begin{equation*}
			\left\llbracket \ \tikzfig{pm_maps/qutrit/X_phase} \ \right\rrbracket = 
			\frac{1}{3} \begin{pmatrix}
				1 + e^{i\alpha} + e^{i\beta} & 1 + \bar{\omega}e^{i\alpha} + {\omega}e^{i\beta} & 1 + {\omega}e^{i\alpha} + \bar{\omega}e^{i\beta} \\
				1 + {\omega}e^{i\alpha} + \bar{\omega}e^{i\beta} & 1 + e^{i\alpha} + e^{i\beta} & 1 + \bar{\omega}e^{i\alpha} + {\omega}e^{i\beta} \\
				1 + \bar{\omega}e^{i\alpha} + {\omega}e^{i\beta} & 1 + {\omega}e^{i\alpha} + \bar{\omega}e^{i\beta} & 1 + e^{i\alpha} + e^{i\beta} \\
			\end{pmatrix}
		\end{equation*}

		So in the simplest case $d=2$ it is fairly straightforward to see that:

		\begin{equation}
			\left\llbracket \ \tikzfig{pm_maps/q2} \ \right\rrbracket \ = \ 
			\frac{1}{2} \begin{pmatrix}
				1 \pm i & 1 \mp i \\
				1 \mp i & 1 \pm i \\
			\end{pmatrix} \ = \ 
			\frac{\sqrt{2}}{2} e^{\mp i \frac{\pi}{4}} \begin{pmatrix}
				\pm i & 1 \\
				1 & \pm i \\
			\end{pmatrix} \ = \ 
			\frac{\sqrt{2}}{2} e^{\mp i \frac{\pi}{4}} \left\llbracket \ \tikzfig{pm_maps/pm} \ \right\rrbracket_{d=2}
		\end{equation}

		The next case $d=3$ is proved similarly:

		\begin{equation}
		\begin{aligned}
				\left\llbracket \ \tikzfig{pm_maps/q3} \ \right\rrbracket
				&= \frac{1}{3} \begin{pmatrix}
					1 + e^{\pm i\frac{2\pi}{3}} + e^{\pm i\frac{2\pi}{3}} & 1 + \bar{\omega}e^{\pm i\frac{2\pi}{3}} + {\omega}e^{\pm i\frac{2\pi}{3}} & 1 + {\omega}e^{\pm i\frac{2\pi}{3}} + \bar{\omega}e^{\pm i\frac{2\pi}{3}} \\
					1 + {\omega}e^{\pm i\frac{2\pi}{3}} + \bar{\omega}e^{\pm i\frac{2\pi}{3}} & 1 + e^{\pm i\frac{2\pi}{3}} + e^{\pm i\frac{2\pi}{3}} & 1 + \bar{\omega}e^{\pm i\frac{2\pi}{3}} + {\omega}e^{\pm i\frac{2\pi}{3}} \\
					1 + \bar{\omega}e^{\pm i\frac{2\pi}{3}} + {\omega}e^{\pm i\frac{2\pi}{3}} & 1 + {\omega}e^{\pm i\frac{2\pi}{3}} + \bar{\omega}e^{\pm i\frac{2\pi}{3}} & 1 + e^{\pm i\frac{2\pi}{3}} + e^{\pm i\frac{2\pi}{3}} \\
				\end{pmatrix} \\
				&= \frac{1}{3} \begin{pmatrix}
					\sqrt{3}e^{\pm i\frac{\pi}{2}} & \sqrt{3}e^{\mp i\frac{\pi}{6}} & \sqrt{3}e^{\mp i\frac{\pi}{6}} \\
					\sqrt{3}e^{\mp i\frac{\pi}{6}} & \sqrt{3}e^{\pm i\frac{\pi}{2}} & \sqrt{3}e^{\mp i\frac{\pi}{6}} \\
					\sqrt{3}e^{\mp i\frac{\pi}{6}} & \sqrt{3}e^{\mp i\frac{\pi}{6}} & \sqrt{3}e^{\pm i\frac{\pi}{2}} \\
				\end{pmatrix} \\
				&= \frac{\sqrt{3}}{3} e^{\mp i\frac{\pi}{6}}\begin{pmatrix}
					e^{\pm i\frac{2\pi}{3}} & 1 & 1 \\
					1 & e^{\pm i\frac{2\pi}{3}} & 1 \\
					1 & 1 & e^{\pm i\frac{2\pi}{3}} \\
				\end{pmatrix} \\
				&= \frac{\sqrt{3}}{3} e^{\mp i\frac{\pi}{6}} \left\llbracket \ \tikzfig{pm_maps/pm} \ \right\rrbracket_{d=3}
			\end{aligned}
		\end{equation}

		For the other qubit case $d=4$ we first note:

		\begin{equation}
			\left\llbracket \ \tikzfig{qubit_hadamard/yellow_box} \ \right\rrbracket = 
			\left\llbracket \ \tikzfig{qubit_hadamard/decomposed} \ \right\rrbracket =
			\left\llbracket \ \qubitZphase{\frac{\pi}{2}} \ \right\rrbracket
			\left\llbracket \ \qubitXphase{\frac{\pi}{2}} \ \right\rrbracket
			\left\llbracket \ \qubitZphase{\frac{\pi}{2}} \ \right\rrbracket = 
			\frac{1}{2\sqrt{2}} \begin{pmatrix}
				1 & 1 \\
				1 & -1 \\
			\end{pmatrix}
		\end{equation}

		Then using the standard interpretation for spiders (Definition \ref{def:qubit_standard_spiders}) we decompose the diagram in such a way that we can apply the standard interpretation:

		\begingroup
			\allowdisplaybreaks
				\begin{align*}
					\left\llbracket \ \tikzfig{pm_maps/q4} \ \right\rrbracket 
					&= \left\llbracket \ \tikzfig{pm_maps/q4/decomposed} \ \right\rrbracket \\
					&= \left(
						\left\llbracket \ \tikzfig{pm_maps/q4/id} \ \right\rrbracket \otimes 
						\left\llbracket \ \tikzfig{pm_maps/q4/pi_compare} \ \right\rrbracket
					\right)
					\left(
						\left\llbracket \ \tikzfig{pm_maps/q4/id} \ \right\rrbracket \otimes 
						\left\llbracket \ \tikzfig{pm_maps/q4/hadamard} \ \right\rrbracket \otimes 
						\left\llbracket \ \tikzfig{pm_maps/q4/id} \ \right\rrbracket 
					\right)
					\left(
						\left\llbracket \ \tikzfig{pm_maps/q4/pi_copy} \ \right\rrbracket \otimes 
						\left\llbracket \ \tikzfig{pm_maps/q4/id} \ \right\rrbracket
					\right) \\
					&= \frac{\sqrt{2}}{8} \begin{pmatrix}
						-1 & 1 & 1 & 1 \\
						1 & -1 & 1 & 1 \\
						1 & 1 & -1 & 1 \\
						1 & 1 & 1 & -1 \\
					\end{pmatrix} \\
					&= \frac{\sqrt{2}}{8} \left\llbracket \ \tikzfig{pm_maps/pm} \ \right\rrbracket_{d=4}
				\end{align*}
		\endgroup
	\end{proof}
\end{proposition}