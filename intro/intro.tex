\section{Introduction}

What to say!?\newline

What's the big idea? What have we shown? That the ZX-calculus can prove non-quantum-computing things? Has this proof essentially been done before (as in, proved via quantum computing) but without the ZX aspect?\newline

Have along the way shown that Aleks' circuit optimisation ideas generalise to the qutrit case.\newline

Potts model usually used for thermodynamics, but has lots of uses - e.g. tumours. Also has other links to maths - e.g. graph colouring.

% Indeed, in the context of quantum computation, additively approximating the Jones polynomial at non-lattice roots of unity is the paradigmatic BQP-complete problem [Aharonov], whereas at the lattice roots of unity the quantum amplitude to which the problem reduces can be computed efficiently. Specifically, the quantum circuits that need to be simulated are stabiliser circuits [Definition?] which are known to be classically tractable [Ref? Kuperberg? Aharonov?]. From the more natural point of view of topological quantum computation, the Jones polynomial at roots of unity corresponds to a quantum amplitude in the fusion space of non-abelian anyons. Computation is performed by initialising, braiding, and [something?] the anyons. Specifically, in the case of SU$(2)_k$ anyons... [Check with Kon about this bit - don't understand it!]\newline

% [Braid-y diagram if a nice one exists, but might well just be confusing - too many knots/braids etc!]\newline
