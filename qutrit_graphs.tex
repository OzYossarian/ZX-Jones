\subsection{Graph-Like Qutrit ZX Diagrams}

% \numbered{Definition}{A \textit{Hadamard edge} (or \textit{H-edge}) is a Hadamard map connecting two spiders. A \textit{Hadamard adjoint edge} (or \textit{$H^\dagger$-edge) }}

[Make all of the following scalar-exact (?)]\newline

We first define a graph-like diagram in the qutrit ZX calculus.

% \numbered{Definition}{A qutrit ZX diagram is \textit{graph-like} when:}
\begin{definition}\label{def:graph_like_qutrit}
	A qutrit ZX-diagram is \textit{graph-like} when: 
	\begin{enumerate}
		\item Every spider is a Z-spider.
		\item Spiders are only connected by Hadamard edges ($H$-edges) or their adjoints ($H^\dagger$-edges).
		\item Every pair of spiders is connected by at most one $H$-edge or $H^\dagger$-edge.
		\item Every input and output is connected to a spider.
		\item Every spider is connected to at most one input or output.
	\end{enumerate}
\end{definition}

For our specific needs, the last two items above will not be relevant, since ZX-diagrams arising from knots will be closed, but we include them to keep the definition consistent with the qubit case. However, note the difference compared to the qubit case: we need not worry about self-loops beacuse the qutrit ZX calculus doesn't define a `plain' cap or cup. But this comes at a cost: spiders in the qutrit case fuse more fussily. Specifically, when two spiders of the same colour are connected by at least one plain edge and at least one $H$- or $H^\dagger$-edge, fusion is not possible. The following equation, which holds with the roles of $H$ and $H^\dagger$ reversed, helps us get around this:

\begin{equation}\label{eq:spiders_reluctant_to_fuse}
	\tikzfig{hadamard_lemmas/spiders_reluctant_to_fuse/1} \ \xeq{\qutritRuleHadamard} \
	\tikzfig{hadamard_lemmas/spiders_reluctant_to_fuse/2} \ \xeq{\qutritRuleId} \
	\tikzfig{hadamard_lemmas/spiders_reluctant_to_fuse/3}
\end{equation}

We will shortly show that every qutrit ZX-diagram is equivalent to a graph-like one, making use of the following lemmas:

\begin{lemma}\label{lem:three_H_edges_vanish}
	The following equation holds in the qutrit ZX-calculus. Moreover, it holds with the roles of $H$ and $H^\dagger$ interchanged:
	\begin{equation}
		\tikzfig{hadamard_lemmas/3_h_edges_vanish/lhs} \quad = \quad \tikzfig{hadamard_lemmas/3_h_edges_vanish/rhs}
	\end{equation}
	\begin{proof}
		It is shown in~\cite[][Lemma 2.8]{qutrit_euler} that the qutrit ZX-calculus satisfies the following `Hopf law':
		\begin{equation}\label{eq:qutrit_hopf}
			\tikzfig{hadamard_lemmas/3_h_edges_vanish/hopf_law}
		\end{equation}
		Therefore we can argue as follows:
		\begin{equation}
			\tikzfig{hadamard_lemmas/3_h_edges_vanish/lhs} \quad \xeq{\qutritRuleHadamard} \quad
			\tikzfig{hadamard_lemmas/3_h_edges_vanish/step_1} \quad \xeq{\qutritRuleColourChange} \quad
			\tikzfig{hadamard_lemmas/3_h_edges_vanish/step_2} \quad \xeq{\eqref{eq:qutrit_hopf}} \quad
			\tikzfig{hadamard_lemmas/3_h_edges_vanish/step_3} \quad \xeq{\eqref{eq:derived_colour_change}} \quad
			\tikzfig{hadamard_lemmas/3_h_edges_vanish/rhs}
		\end{equation}
	\end{proof}
\end{lemma}

\begin{lemma}\label{lem:H_edges_qutrit} 
	The following two equations hold in the qutrit ZX-calculus. Moreover, they hold with the roles of $H$ and $H^\dagger$ interchanged:
	\begin{equation}
		\tikzfig{hadamard_lemmas/2_h_edges_flip}
		\hspace{75pt}
		\tikzfig{hadamard_lemmas/h_edge_and_adjoint_permute}
	\end{equation}
	\begin{proof}
		This is Lemma 3.4 in\ \cite{qutrit_euler}.
	\end{proof}
\end{lemma}

As we will formalise later, the lemmas above say that we can think of Hadamard edges as $1$-weighted edges and their adjoints as $2$-weighted edges, then work modulo $3$, since every triple of parallel edges disappears. This motivates defining `parametrised' Hadamard gates, which will come in use later:

\begin{equation}
	\tikzfig{hadamard_lemmas/parametrised/0}
	\hspace{75pt}
	\tikzfig{hadamard_lemmas/parametrised/1}
	\hspace{75pt}
	\tikzfig{hadamard_lemmas/parametrised/2}
\end{equation}

The first equation above just says that a `$0$-Hadamard edge' is in fact the empty diagram, and not an edge at all. Where the previous lemmas relate single $H$- and $H^\dagger$-boxes across multiple edges, the next relates multiple $H$- and $H^\dagger$- boxes on single edges.

\begin{lemma}\label{lem:H_boxes_qutrit} 
	The following three equations hold in the qutrit ZX calculus. Moreover, they hold with the roles of $H$ and $H^\dagger$ interchanged:
	\begin{equation}
		\tikzfig{hadamard_lemmas/4_h_boxes_vanish}
		\hspace{75pt}
		\tikzfig{hadamard_lemmas/3_h_boxes_flip}
		\hspace{75pt}
		\tikzfig{hadamard_lemmas/2_h_boxes_separate}
	\end{equation}
	\begin{proof}
		For the first equation, turn the top two $H$-boxes into $H^\dagger$-boxes via $\qutritRuleSnake$, then cancel twice via $\qutritRuleHadamard$. Similarly, for the second equation turn the top two $H$-boxes into $H^\dagger$-boxes via $\qutritRuleSnake$, then cancel once via $\qutritRuleHadamard$, leaving a single $H^\dagger$-box. The third equation is just a simple application of $\qutritRuleId$. Analogous reasoning applies when the roles of $H$ and $H^\dagger$ are swapped.
	\end{proof}
\end{lemma}

Again, intuitively we can think of Hadamard boxes of having value $1$ and their adjoints $-1$ and then work modulo $4$.

\begin{corollary}\label{prop:every_diagram_is_graph_like_qutrit}
	Every qutrit ZX diagram is equivalent to one that is graph-like.
	\begin{proof}
		First use the colour change rule to turn all X-spiders into Z-spiders. Then use Lemma \ref{lem:H_boxes_qutrit} to remove excess $H$- and $H^\dagger$-boxes, inserting a spider between any remaining consecutive pair of such boxes, so that all spiders are connected only by plain edges, $H$-edges or $H^\dagger$-edges. Fuse together as many as possible, and apply \eqref{eq:spiders_reluctant_to_fuse} where fusion is not possible, so that no plain edge connects two spiders. Apply Lemmas \ref{lem:three_H_edges_vanish} and \ref{lem:H_edges_qutrit} to all connected pairs of spiders until at most one $H$- or $H^\dagger$-edge remains between them. Finally, to ensure every input and output is connected to a spider and every spider is connected to at most one input or output, we can use $\qutritRuleHadamard$ and $\qutritRuleId$ to add a few spiders, $H$- and $H^\dagger$-boxes as needed: 
		\begin{equation}
			\tikzfig{is_graph_like/plain_input_output_wire}
			\hspace{75pt}
			\tikzfig{is_graph_like/input_connected_to_hadamard}
			\hspace{75pt}
			\tikzfig{is_graph_like/multiple_inputs_connected_to_one_spider}
		\end{equation}
	\end{proof}
\end{corollary}

\begin{definition}\label{def:graph_state_qutrit}
	A graph-like qutrit ZX diagram is a \textit{graph state} when every spider has zero phase (top and bottom) and is connected to an output. 
\end{definition}

A graph state is described fully by its underlying multigraph, or equivalently by an adjacency matrix, where edges take weights in $\mathbb{Z}_3$\ \cite[][Lemma 4.2]{harny_completeness}. Nodes correspond to phaseless green spiders, edges of weight $1$ correspond to Hadamard edges, and edges of weight $2$ correspond to $H^\dagger$ edges. As in the qubit case, graph states admit a local complementation operation\ \cite[][Definition 2.6]{harny_completeness}, though the effect is now slightly more complicated. We'll give the intuition after the formal definition:

\begin{definition}\label{def:local_complementation_qutrit}
	Given $a \in \mathbb{Z}_3$ and a graph state $G$ with adjacency matrix $W = (w_{i,j})$, the \textit{$a$-local complentation} at node $k$ is the new graph state $G *_a i$, whose adjacency matrix $W' = (w'_{i,j})$ given by:
	\begin{equation}
		w'_{i,j} = w_{i,j} + aw_{i,k}w_{j,k}
	\end{equation}
\end{definition}

So only those edges between neighbours of node $k$ are affected, but rather than just having their weight increased by $1$ (modulo $2$) as in the qubit case, the increase in weight also depends on the weights of the edges from $i$ and $j$ to $k$. As always, this is best seen graphically. We reintroduce the blue dashed line notation for Hadamard edges, and now also use purple dashed lines for $H^\dagger$-edges:

\begin{equation}\label{eq:qutrit_dashed_lines}
	\tikzfig{hadamard_lemmas/blue_dashed_line}
	\hspace{75pt}
	\tikzfig{hadamard_lemmas/purple_dashed_line}
\end{equation} 

For two nodes $i$ and $j$ both connected to $k$ by the same colour edge, $a$-local complementation at $k$ increases weight $w_{i,j}$ by $a$. If instead $i$ and $j$ are connected to $k$ by edges of different colour, $a$-local complementation at $k$ decreases $w_{i,j}$ by $a$. We show a fragment of a ZX-diagram below under the effect of this operation:

\begin{equation}
	\tikzfig{a_local_comp/same}
	\hspace{75pt}
	\tikzfig{a_local_comp/different}
\end{equation}

But the fragment above doesn't give the full picture. As in the qubit case, local complementation gives an equality up to introducing some single qubit phase gates on the outputs.

\begin{theorem}\label{thm:local_comp_equality}
	Given $a \in \mathbb{Z}_3$ and a graph state $(G, W)$ containing a node $k$, let $N(k)$ denote the neighbours of $k$ - that is, nodes $i$ with weight $w_{i,k} \in \{1, 2\}$. Then the following equality holds:
	\ctikzfig{graph_state/local_comp}
	\begin{proof}
		This is Theorem 4.4 and Corollary 4.5 in\ \cite[][]{harny_completeness}.
	\end{proof}
\end{theorem}

Composing local complementations gives a local pivot operation.

\begin{definition}\label{def:local_pivot_qutrit}
	Given $a,b,c \in \mathbb{Z}_3$ and a graph state $G$ containing nodes $i$ and $j$, the \textit{$(a,b,c)$-local pivot} along $ij$ is the new graph state $G \wedge_{(a,b,c)} ij \defeq ((G *_a i) *_b j) *_c i$. 
\end{definition}

This again results in an equality, up to introducing some extra gates on outputs. Here we shall only consider an $(a,-a,a)$-local pivot along an edge $ij$ of non-zero weight, for $a \in \{1, 2\}$. We will call this a \textit{proper $a$-local pivot} along $ij$, and denote it $G \wedge_a ij$.

\begin{theorem}\label{thm:local_pivot_equality}
	Given $a \in \mathbb{Z}_3$ and a graph state $(G, W)$ containing connected nodes $i$ and $j$, define the following:
	\begin{itemize}
		\item $N_{=}(i, j) \defeq \left\{k \in N(i) \cap N(j) \mid w_{k,i} = w_{k,j} \right\}$
		\item $N_{\neq}(i, j) \defeq \left\{k \in N(i) \cap N(j) \mid w_{k,i} \neq w_{k,j} \right\}$
	\end{itemize} 
	Then the following equation relates $G$ and its proper $a$-local pivot along $ij$:
	\ctikzfig{graph_state/proper_local_pivot}

	\begin{proof}
		See Appendix \ref{thm:local_pivot_equality_appendix}.
	\end{proof}
\end{theorem}

These two operations are again the drivers behind the simplification procedure. We classify spiders into three families:

\begin{equation*}
	\mathcal{M} = \left\{\qutritZspider{0}{0}, \qutritZspider{1}{2}, \qutritZspider{2}{1}\right\},
	\hspace{10pt}
	\mathcal{N} = \left\{\qutritZspider{0}{1}, \qutritZspider{1}{0}, \qutritZspider{0}{2}, \qutritZspider{2}{0}\right\},
	\hspace{10pt}
	\mathcal{P} = \left\{\qutritZspider{1}{1}, \qutritZspider{2}{2}\right\}.
\end{equation*}

 exactly as in~\cite[][Theorem 3.1]{harny_completeness}. We call a spider in a graph-like ZX-diagram \textit{interior} if it isn't connected to an input or output (so for our use case all spiders are interior). Given any graph-like ZX-diagram, we will show that we can eliminate pairs of connected interior \Mspiders\ by local pivoting, and standalone interior $\mathcal{N}$- and \Pspiders\ by local complementation.

\begin{theorem}\label{thm:eliminate_M_spiders}
	Given any graph-like ZX-diagram containing two interior \Mspiders\ $i$ and $j$ connected by edge $ij$ of weight $w_{i,j} \eqdef w \in \{1,2\}$, suppose we perform a proper $\pm w$-local pivot along $ij$. Then the new ZX-diagram is related to the old one by the equality:
	
	\ctikzfig{eliminate_M_spiders/theorem_LHS}
	\ctikzfig{eliminate_M_spiders/theorem_RHS}
	
	where all changes to weights of edges where neither endpoint is $i$ or $j$ are omitted. Furthermore, in order to save space, each node with phase \qutritZphase{a_x^y}{b_x^y} is representative of \textit{all} nodes connected to $i$ by an $x$-weighted edge and to $j$ by a $y$-weighted edge.

	\begin{proof}
		See \ref{thm:eliminate_M_spiders_appendix} in the Appendix.
	\end{proof}
\end{theorem}

\begin{theorem}\label{thm:eliminate_N_spiders}
	Given any graph-like ZX-diagram containing an interior \Nspider\ $k$ with phase \qutritZphase{0}{n} for $n \in \{1,2\}$, suppose we perform a $(-n)$-local complementation at $k$. Then the new ZX-diagram is related to the old one by the equality:

	\begin{equation*}
		\tikzfig{eliminate_N_spiders/0_n/step_1} \quad = \quad \tikzfig{eliminate_N_spiders/0_n/step_9}
	\end{equation*}

	where all changes to weights of edges where neither endpoint is $k$ are omitted. If instead $k$ has phase \qutritZphase{n}{0} for $n \in \{1,2\}$, suppose we perform the same $(-n)$-local complementation at $k$. Then the equality relating the new and old diagrams becomes:

	% Spiders with phases \qutritZphase{a_1}{b_1} ... \qutritZphase{a_r}{b_r} are all the neighbours of $k$ connected by a $1$-weighted (blue) edge, while spider with phases \qutritZphase{c_1}{d_1} ... \qutritZphase{c_s}{d_s} are all the neighbours of $k$ connected by a $2$-weighted (purple) edge.\newline

	\begin{equation*}
		\tikzfig{eliminate_N_spiders/n_0/step_1} \quad = \quad \tikzfig{eliminate_N_spiders/n_0/step_9}
	\end{equation*}

	\begin{proof}
		See Appendix \ref{thm:eliminate_N_spiders_appendix}.
	\end{proof}
\end{theorem}

\begin{theorem}\label{thm:eliminate_P_spiders}
	Given any graph-like ZX-diagram containing an interior \Pspider\ $k$ with phase \qutritZphase{p}{p} for $p \in \{1,2\}$, suppose we perform a $p$-local complementation at $k$. Then the new ZX-diagram is related to the old one by the equality:

	\begin{equation*}
		\tikzfig{eliminate_P_spiders/step_1} \quad = \quad \tikzfig{eliminate_P_spiders/step_9}
	\end{equation*}

	where all changes to weights of edges where neither endpoint is $k$ are omitted.

	\begin{proof}
		See Appendix \ref{thm:eliminate_P_spiders_appendix}.
	\end{proof} 

	% Spiders with phases \qutritZphase{a_1}{b_1} ... \qutritZphase{a_r}{b_r} are all the neighbours of $k$ connected by a $1$-weighted (blue) edge, while spider with phases \qutritZphase{c_1}{d_1} ... \qutritZphase{c_s}{d_s} are all the neighbours of $k$ connected by a $2$-weighted (purple) edge.\newline
\end{theorem}

We can now combine these three elimination theorems into an algorithm for efficiently simplifying a closed graph-like ZX-diagram. First note that after applying any one of the three elimination theorems to such a diagram we may end up with a state that is no longer graph-like. Fortunately the only way in which this can happen is if two spiders end up being connected by multiple $H$- or $H^\dagger$-edges, and we have shown via Lemmas \ref{lem:three_H_edges_vanish} and \ref{lem:H_edges_qutrit} that these can always be reduced to just one edge.

\begin{theorem}\label{thm:simplification_algorithm_works}
	Given any closed graph-like ZX-diagram, the following algorithm will always terminate after a finite number of steps, returning an equivalent graph-like ZX-diagram with no \Nspiders, \Pspiders, or adjacent pairs of \Mspiders. Repeat the steps below until no rule matches. After each step, apply Lemmas \ref{lem:three_H_edges_vanish} and \ref{lem:H_edges_qutrit} as needed until the resulting diagram is graph-like:
	\begin{enumerate}
		\item Eliminate an \Nspider\ via Theorem~\ref{thm:eliminate_N_spiders}.
		\item Eliminate a \Pspider\ via Theorem~\ref{thm:eliminate_P_spiders}.
		\item Eliminate two adjacent \Mspiders\ via Theorem~\ref{thm:eliminate_M_spiders}.
	\end{enumerate}
	\begin{proof}
		At every step the total number of spiders decreases by at least one, so since we start with a finite diagram the algorithm terminates after a finite number of steps. By construction, when it does so we are left with an equivalent graph-like ZX-diagram with no \Nspiders, \Pspiders, or adjacent pairs of \Mspiders.
	\end{proof}
\end{theorem}

\begin{corollary}\label{cor:stabilizer_simplification_algorithm_works}
	In particular, if we start with a stabilizer diagram, we can eliminate all but perhaps one \Mspider, depending on whether the initial number of \Mspiders\ was odd or even. 
	\begin{proof}
		No step introduces any non-stabilizer phases.
	\end{proof}
\end{corollary}

The algorithm above could be extended to graph-like diagrams with inputs or outputs as in \cite[][Theorem 5.4]{graph_theoretic_simplification}, but since for our purposes we don't need to do so, we have not gone to the trouble.